% ----
% COMP1204 CW1 Report Document
% ----
\documentclass[]{article}
\usepackage{courier}
\usepackage{minted}
\usepackage{graphicx}

\usepackage[margin=1in]{geometry}

\setlength\parindent{0pt}

\title{COMP1204: Data Management \\ Coursework Two: SQL }
\author{Xiaoke Li \\ 31951473}


\begin{document}
\maketitle
\newpage

\section{The Relational Model}

\subsection{EX1}

Relation: 


dataset(dateRep, day, month, year, cases, deaths,\\ countriesAndTerritories, geoId, countryterritoryCode,\\ popData2020, continentExp)\\ \\

attributes and types
\begin{enumerate}
    \item dateRep: TEXT
    \item day: INTEGER
    \item month: INTEGER
    \item year: INTEGER
    \item cases: INTEGER
    \item deaths: INTEGER
    \item countriesAndTerritories: TEXT
    \item geoId: TEXT
    \item countryterritoryCode: TEXT
    \item popData2020: INTEGER
    \item continentExp: TEXT
\end{enumerate}

\subsection{EX2}
Minimal set of FDs:\\
\textbf{(Any two country in the world may or may not have the same population, thus here let assume they will have same value)}
\begin{enumerate}
    \item  dateRep $\to$ day
    \item  dateRep $\to$ month
    \item  dateRep $\to$ year
    \item (day,month,year) $\to$ dateRep
    \item geoId $\to$ countriesAndTerritories
    \item geoId $\to$ countryterritoryCode
    \item geoId $\to$ continentExp
    \item geoId $\to$ popData2020
    \item (dateRep,geoId)  $\to$ cases
    \item (dateRep,geoId)  $\to$ deaths\\
    \textbf{Contrary}
    \item countriesAndTerritories $\to$ geoId
    \item countryterritoryCode $\to$ geoId
\end{enumerate}

\subsection{EX3}
\textbf{list all the potential candidate keys}
\begin{enumerate}
    \item  dateRep,geoId
    \item  dateRep,countriesAndTerritories
    \item  dateRep,countryterritoryCode
\end{enumerate}

\subsection{EX4}
My primary key: \\
\textbf{ dateRep,geoId }\\
Reason: 
\begin{enumerate}
    \item  It is candidate key
    \item  geoId is short and Easy to distinguish
    \item  Each geoId is unique
\end{enumerate}

\newpage
\section{Normalisation}
\subsection{EX5}
\textbf{ Partial-key dependencies: }
\begin{itemize}
    \item dateRep$\to$ day
    \item dateRep$\to$ month
    \item dateRep$\to$ year
    \item geoId $\to$ countryterritoryCode
    \item geoId $\to$ continentExp
    \item geoId $\to$ popData2020
    \item geoId $\to$ continentExp
\end{itemize}
Then we divide it into 3 tables
\begin{enumerate}
    \item dateRep,geoId $\to$ cases, deaths
    \item dateRep $\to$ day, month, year
    \item geoId $\to$ countriesAndTerritories, countryterritoryCode, popData2020, continentExp
\end{enumerate}

\subsection{EX6}
\textbf{2NF Normal Form}
\begin{itemize}
    \item Cases(\underline{ dateRep,geoId}, cases, deaths)
    \item Date(\underline{ dateRep}, day, month, year)
    \item Country(\underline{ geoId}, countriesAndTerritories, countryterritoryCode, popData2020, continentExp)
\end{itemize}
The primary keys for their respective tables are:\\
-Date - dateRep \\
-Country - geoId \\
-Cases - dateRep, geoId \\

The foreign keys for their respective tables are:\\
-Date - None \\
-Country - None    \\
-Cases - dateRep, geoId   \\

\subsection{EX7}
There are no more transitive dependencies in those relations.


\subsection{EX8}
The relations are the same as which is in 2NF, because there are no transitive dependencies.
All the non key attributes are either determined by key.

\subsection{EX9}
It is a BCNF.
Every determinant is a candidate key!


\newpage
\section{Modelling}

\subsection{EX10}
\begin{minted}[linenos]{bash}
sqlite3 coronavirus.db 
.open coronavirus.db // create new database
.mode csv 
//Then use DataGrip -> Import Data From File(Automatic import)
sqlite3 coronavirus.db .dump > database.sql
\end{minted}

\subsection{EX11}
\begin{minted}[linenos]{bash}
create table Date
(
    dateRep Text    not null
        primary key,
    day     INTEGER not null,
    month   INTEGER not null,
    year    INTEGER not null
);

create table Country
(
    geoId                   TEXT    not null
        constraint Country_pk
            primary key,
    countriesAndTerritories TEXT    not null,
    countryterritoryCode    TEXT    not null,
    popData2020             integer not null,
    continentExp            TEXT    not null
);

create table Cases
(
    dateRep TEXT not null,
    geoId   TEXT not null,
    deaths  integer,
    cases   integer,
    constraint Cases_pk
        primary key (dateRep, geoId)
);

\end{minted}

\subsection{EX12}
\begin{minted}[linenos]{bash}
INSERT INTO Date (dateRep, day, month, year)
SELECT DISTINCT dateRep, day, month, year
FROM dataset;

INSERT INTO Country (geoId, countryterritoryCode, continentExp, countriesAndTerritories, popData2020)
SELECT DISTINCT geoId, countryterritoryCode, continentExp, countriesAndTerritories, popData2020
FROM dataset;

INSERT INTO Cases (dateRep, geoId, cases, deaths)
SELECT DISTINCT dateRep, geoId, cases, deaths
FROM dataset;
\end{minted}

\subsection{EX13}
\begin{minted}[linenos]{bash}
sqlite3 coronavirus.db < dataset.sql
sqlite3 coronavirus.db < ex11.sql
sqlite3 coronavirus.db < ex12.sql  
// use test.db to replace coronavirus.db
\end{minted}


\section{Querying}
\subsection{EX14}
\begin{minted}[linenos]{bash}
SELECT sum(cases) AS total_cases, sum(deaths) AS total_deaths
FROM Cases;
\end{minted}

\subsection{EX15}
\begin{minted}[linenos]{bash}
SELECT Cases.dateRep,cases
FROM Cases
INNER JOIN Date ON Date.dateRep = Cases.dateRep
WHERE Cases.geoId = 'UK'
ORDER BY year,month,day ASC;
\end{minted}

\subsection{EX16}
\begin{minted}[linenos]{bash}
select countriesAndTerritories AS country, Cases.dateRep AS date,
       Cases.cases, Cases.deaths
from Cases
natural join Country
natural join Date
GROUP BY country,dateRep
ORDER BY countriesAndTerritories ASC ,year,month,day ASC
\end{minted}


\subsection{EX17}
\begin{minted}[linenos]{bash}
SELECT Country.countriesAndTerritories ,
       round((sum(cases) * 1.0 / popData2020) * 100, 2) AS "cases %",
       round((sum(deaths) * 1.0 / popData2020) * 100, 2) AS "deaths %"
FROM Cases
NATURAL JOIN Country
GROUP BY  countriesAndTerritories,popData2020;
\end{minted}

\subsection{EX18}
\begin{minted}[linenos]{bash}
SELECT countriesAndTerritories AS "country",
       ((CAST(sum(deaths) AS REAL) * 100) / sum(cases)) AS "deaths % cases"
FROM Cases
INNER JOIN Country ON Cases.geoId = Country.geoId
GROUP BY country
ORDER BY "deaths % cases" DESC
LIMIT 10;
\end{minted}

\subsection{EX19}
\begin{minted}[linenos]{bash}
SELECT Cases.dateRep AS "date",
       sum(deaths) OVER win1 AS "cumulative UK deaths",
       sum(cases) OVER win1 AS "cumulative UK cases"
FROM Cases
NATURAL JOIN Date
WHERE geoid='UK'
WINDOW win1 AS (ROWS BETWEEN UNBOUNDED PRECEDING AND CURRENT ROW)
ORDER BY year, month, day ASC ;
\end{minted}

\end{document}