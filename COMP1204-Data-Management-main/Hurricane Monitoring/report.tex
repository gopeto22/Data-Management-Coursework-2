% ----
% COMP1204 CW1 Report Document
% ----
\documentclass[]{article}
\usepackage{courier}
\usepackage{minted}
\usepackage{graphicx}

\usepackage[margin=1in]{geometry}

\setlength\parindent{0pt}

\title{COMP1204: Data Management \\ Coursework One: Hurricane Monitoring }
\author{Georgi Iliev}

\begin{document}
\maketitle

\section{Introduction}
This work aims to extract useful storm data from a set of hurricane reports in KML format.
The structure of tags in these reports resemble that of XML files.
Instead of using an XML parser, the script uses regular expressions and
the Unix utility sed extensively to filter and group data of individual storms.

{\parskip=7pt
The extracted data is exported in CSV format. Storm plots are generated as PNG files
from the CSV output using the provided create\_map\_plot.sh script.}

\section{Create CSV Script}

\begin{minted}[linenos]{bash}
#!/bin/bash

# 1 Set 2 varialbes to extract data from KML,and save as CSV

kml_input=$1
csv_output=$2

#   Set the first line of output file
echo "Timestamp,Latitude,Longitude,MinSeaLevelPressure,MaxIntensity" > $2

# 2 Collect the 5 required data
# 	Cut off useless parts then add unit at the end of each line

#   grep timestamp
grep "<dtg>" $1 | cut -d ">" -f 2 | cut -d "<" -f 1 > ts.txt

#   grep latitude
grep "<lat>" $1 | cut -d ">" -f 2 | cut -d "<" -f 1 \
    |  sed "s/$/& N/g" >lat.txt

#   grep longitude
grep "<lon>" $1 | cut -d ">" -f 2 | cut -d "<" -f 1 \
    | sed "s/$/& W/g" > lon.txt

#   grep max intensity
grep "<intensity>" $1 | cut -d ">" -f 2 | cut -d "<" -f 1 \
    | sed "s/$/& knots/g" > inten.txt

#   grep min sealevel pressure
grep "<minSea.*>" $1 | cut -d ">" -f 2 | cut -d "<" -f 1 \
    | sed "s/$/& mb/g" > press.txt

# 3 Paste data in order, output final data as a CSV file
paste -d ',' ts.txt lat.txt lon.txt press.txt inten.txt >>$2
\end{minted}

\section{Storm Plots}
\begin{figure}[ht]
    \centering
    \includegraphics[scale=0.8]{al102020.png}
    \caption{Generated from al102020.kml}
\end{figure}
\begin{figure}[ht]
    \centering
    \includegraphics[scale=0.8]{al112020.png}
    \caption{Generated from al112020.kml}
\end{figure}
\begin{figure}[ht]
    \centering
    \includegraphics[scale=0.8]{al122020.png}
    \caption{Generated from al122020.kml}
\end{figure}

\end{document}
